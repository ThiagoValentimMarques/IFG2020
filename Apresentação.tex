%--------------------------------------------%
% Template Beamer para Apresentações da UFRN %
% by alcemygvseverino@gmail.com              %
% Baseado em MIT Beamer Template			 %
% versao 1.1								 %
% Atualizado em 14/05/2016					 %
%--------------------------------------------%

\documentclass[xcolor=table,t]{beamer}
% Para alterar a linguagem do documento
\usepackage[portuges]{babel}
% Para aceitar caracteres especias deretamente do teclado
\usepackage[utf8]{inputenc}
% Para seguir as normas da ABNT de citacao e referencias
\usepackage[alf]{abntex2cite}
% Para incluir figuras
\usepackage{graphicx}
% Para melhor ajuste da posisao das figuras
\usepackage{float}
% Para ajustar as dimensoes do layout da pagina
\usepackage{geometry}
% Para formatar estrutura e informacoes de formulas matematicas
\usepackage{amsmath}
% Para incluir simbolos especiais em formulas matematicas
\usepackage{amssymb}
% Para incluir links nas referencias
\usepackage{url}
% Para incluir paginas de documentos .pdf externos
\usepackage{pgfpages}
% Para ajustar o estilo dos contadores
\usepackage{enumerate}
% Para modificar a cor do texto
\usepackage{color}
% Para incluir condicoes
\usepackage{ifthen}
% Para colocar legendas em algo que nao e float
\usepackage{capt-of}
% Para usar o texto justificado
\usepackage{ragged2e}
% Para definir o tema do slide
\usetheme{Berlin}
% Uso do \verbatim
\usepackage{verbatim}
\usepackage{float}
%Figuras lado a lado \begin{figure}[H]
% Para difinir cores e background
\usetheme{Madrid}
\usecolortheme{ufrn}
% Para numerar as figuras
\setbeamertemplate{caption}[numbered]
%Para retirar a hifenação
\usepackage{hyphenat}
\tolerance=1 
\emergencystretch=\maxdimen 
\hyphenpenalty=10000 
\hbadness=10000
\hyphenchar\font=-1 
\sloppy 
%Parágrafo sem identação
\setlength{\parindent}{1.0 cm} 
%margens
\setbeamersize{text margin left=7mm,text margin right=7mm}

\DefineNamedColor{named}{roxobb}{RGB}{99,26,43} %roxo batebola debate
\DefineNamedColor{named}{verdebb}{RGB}{186,229,114} %verde batebola debate
 

% Título
    \title[\href{https://www.youtube.com/watch?v=BkFAt3UnAMM}{11ª SEMAT - IFG Câmpus Goiânia}]{Análise exploratória de dados epidemiológicos}
	
% Data

\date{\footnotesize{1 de julho de 2020}}
% Autores
\author[\href{http://lattes.cnpq.br/0803375312960649}{Thiago Valentim Marques}]{
	Thiago Valentim Marques \vspace{-0.5 cm}}
	
% Instituto
\institute[\href{www.thiagovalentim.me}{www.thiagovalentim.me}]{\scriptsize{Instituto Federal de Educação, Ciência e Tecnologia do Rio Grande do Norte (IFRN)\\
		Campus Natal - Zona Norte} \\
		\vspace{0.6 cm}
		

	\includegraphics[width=0.18\textwidth]{figuras/semat} \hspace{0.8 cm}
	\includegraphics[width=0.17\textwidth]{figuras/ifg}
	\vspace{0.5 cm}
}

% Logo do canto inferior direito
\pgfdeclareimage[height=1.0cm]{logo_UFRN}{figuras/logo_UFRN}
\logo{
	\vspace*{0.02cm}
	\pgfuseimage{logo_UFRN}
	\hspace*{-0.08cm}}


\begin{document}


\frame{\titlepage}

\section[]{}
\begin{frame}{Sumário}
	\tableofcontents
\end{frame}

\section{Quem sou eu?}
\begin{frame}{Sumário}
\tableofcontents[currentsection]
\end{frame}

\begin{frame}[c]{Quem sou eu?}
\vspace{-0.3 cm}
\begin{figure}
	\centering
	\hspace{-1.2 cm}
	\includegraphics[width=0.64 \linewidth]{figuras/linha}
\end{figure}
\vspace{-0.4 cm}
\tiny
Elaborado no site \href{https://infograph.venngage.com/templates/recommended}{\textcolor{blue}{venngage.com}}.
\end{frame}

\section{Análise exploratória de dados}
\begin{frame}{Sumário}
\tableofcontents[currentsection]
\end{frame}

\begin{frame}[c]{Análise exploratória de dados}
\begin{itemize}
	\justifying
	\item[$\checkmark$] O primeiro passo em qualquer análise de dados é a análise exploratória; 
	\vspace{0.6 cm}
	\item[$\checkmark$] É possível gerar ideias e \textit{insights} apenas observando os dados plotados;
	\vspace{0.6 cm}
	\item[$\checkmark$] Promovida pelo matemático norte-americano \href{https://pt.wikipedia.org/wiki/John_Tukey}{\textcolor{blue}{John Tukey}} (1915-2000), que incentivava os estatísticos a explorar os dados e possivelmente formular hipóteses que poderiam levar a novas coletas de dados e experimentos.
\end{itemize}
\end{frame}

\begin{frame}[c]{Análise exploratória de dados}
\hspace{-1.50 cm}\begin{minipage}[c]{4.0 cm}
	\centering
	\includegraphics[width=0.9 \linewidth]{figuras/rstudio} 
\end{minipage}
\hspace{0.3 cm}
\begin{minipage}[r]{7 cm}
   \small \textbf{Software:} R  \\
	\textbf{IDE:} RStudio \\ 
	\textbf{Site do R:} \href{https://www.r-project.org/}{https://www.r-project.org/} \\
	\textbf{Site do RStudio:} \href{https://rstudio.com/}{https://rstudio.com/} \\
\end{minipage}

\vspace{0.5 cm}
\includegraphics[width=0.9 \linewidth]{figuras/tidyverse} 
\end{frame}

\begin{frame}[c, fragile]{Análise exploratória de dados (exemplo no R)}
\justifying
Dados extraídos da revista norte-americana Motor Trend, de 1974, que incluem consumo de combustível e 10 aspectos do design e desempenho de 32 automóveis.
\vspace{0.4 cm}
\scriptsize
\begin{verbatim}
head(mtcars)

##                    mpg cyl disp  hp drat    wt  qsec vs am gear carb
## Mazda RX4         21.0   6  160 110 3.90 2.620 16.46  0  1    4    4
## Mazda RX4 Wag     21.0   6  160 110 3.90 2.875 17.02  0  1    4    4
## Datsun 710        22.8   4  108  93 3.85 2.320 18.61  1  1    4    1
## Hornet 4 Drive    21.4   6  258 110 3.08 3.215 19.44  1  0    3    1
## Hornet Sportabout 18.7   8  360 175 3.15 3.440 17.02  0  0    3    2
## Valiant           18.1   6  225 105 2.76 3.460 20.22  1  0    3    1
\end{verbatim}
\end{frame}

\begin{frame}[t, fragile]{Análise exploratória de dados (exemplo no R)}
\vspace{-0.3 cm}
\scriptsize
\begin{verbatim}
# Carregando o pacote tidyverse
require(tidyverse)

# Transformando a variável cyl em fator
mtcars <- mtcars %>% 
mutate(cyl = as.factor(cyl))

# Gráficos de dispersão

ggplot(data=mtcars,aes(x=wt,y=mpg))+geom_point()
ggplot(data=mtcars,aes(x=wt,y=mpg,color=cyl))+geom_point()

\end{verbatim}
\vspace{-0.7 cm}
\begin{figure}[htb]
	\begin{minipage}[t]{.45\textwidth}
		\centering
		\includegraphics[width=1.00 \linewidth]{figuras/plot1}
	\end{minipage}
	\begin{minipage}[t]{.45\textwidth}
		\centering
		\includegraphics[width=1.00 \linewidth]{figuras/plot2}
	\end{minipage}  
\end{figure}
\end{frame}

\begin{frame}[t, fragile]{Análise exploratória de dados (exemplo no R)}
\vspace{-0.3 cm}
\scriptsize
\begin{verbatim}
ggplot(data=mtcars,aes(x=wt,y=mpg,color=cyl))+
geom_point()+theme_update()+
labs(title="Gráfico de dispersão do peso versus consumo",
caption="Fonte: Motor Trend        Autor: Thiago Valentim",
x = "Peso (1000 lb)",
y = "Consumo (mi/gal)",
colour = "Cilindros")
\end{verbatim}
\vspace{-0.2 cm}
\begin{figure}[htb]
		\centering
		\includegraphics[width=0.75 \linewidth]{figuras/plot3}
\end{figure}
\end{frame}

\begin{frame}[c]{Download de \textit{datasets}}
\begin{itemize}
\justifying
\item[$\checkmark$] \normalsize{Portal Brasileiro de Dados Abertos} \newline 
\href{http://dados.gov.br/}{\beamergotobutton{Link}}
\vspace{0.1 cm}
\item[$\checkmark$] \normalsize{Instituto Brasileiro de Geografia e Estatística (IBGE)} \newline
\href{http://downloads.ibge.gov.br/}{\beamergotobutton{Link}}
\vspace{0.1 cm}
\item[$\checkmark$] \normalsize{Banco Central do Brasil} \newline 		\href{https://www3.bcb.gov.br/sgspub/localizarseries/localizarSeries.do?method=prepararTelaLocalizarSeries}{\beamergotobutton{Link}}
\vspace{0.1 cm}
\item[$\checkmark$] \normalsize{Open Data SUS} \newline
\href{https://opendatasus.saude.gov.br/}{\beamergotobutton{Link}}
\vspace{0.1 cm}
\item[$\checkmark$]  \normalsize{Instituto Nacional de Meteorologia (INMET)} \newline 
\href{http://www.inmet.gov.br/portal/index.php?r=bdmep/bdmep}{\beamergotobutton{Link}}
\vspace{0.1 cm}
\item[$\checkmark$]  \normalsize{Instituto Nacional de Águas (ANA) - Evapotranspiração} \newline 
\href{https://ssebop.users.earthengine.app/view/ssebop-br}{\beamergotobutton{Link}}
\end{itemize}
\end{frame}

\begin{frame}[c]{Download de \textit{datasets}}
\begin{itemize}
\justifying
\item[$\checkmark$] \normalsize{Youtube Datasets} \newline
\href{https://research.google.com/youtube8m/}{\beamergotobutton{Link}}
\vspace{0.1 cm}
\item[$\checkmark$] \normalsize{UCI Machine Learning Repository} \newline
\href{https://archive.ics.uci.edu/ml/datasets.php}{\beamergotobutton{Link}}
\vspace{0.1 cm}
\item[$\checkmark$] \normalsize{Kaggle Datasets} \newline
\href{http://kaggle.com/}{\beamergotobutton{Link}}
\vspace{0.1 cm}
\item[$\checkmark$] \normalsize{US Government Datasets} \newline
\href{http://catalog.data.gov/dataset}{\beamergotobutton{Link}}
\vspace{0.1 cm}
\item[$\checkmark$] \normalsize{European Union (EU) Open Data Portal} \newline
\href{http://data.europa.eu/euodp/en/data/}{\beamergotobutton{Link}}
\end{itemize}
\end{frame}

\begin{frame}[c]{Charge}
\centering
\includegraphics[width=0.60\linewidth]{figuras/sindados}
\end{frame}

\begin{frame}[c]{Marcus Nunes'Blog (sugestão de blog)}
\centering
\includegraphics[width=0.9\linewidth]{figuras/mnunes}
\begin{flushleft}
	\footnotesize
	\href{https://marcusnunes.me/}{https://marcusnunes.me/}
\end{flushleft}
\end{frame}

\begin{frame}[c]{Aplicações Shiny - UFRN}
\centering
\includegraphics[width=1.0\linewidth]{figuras/tvalentim2}
\begin{flushleft}
	\footnotesize
	\href{http://shiny.estatistica.ccet.ufrn.br/}{http://shiny.estatistica.ccet.ufrn.br/}
\end{flushleft}
\end{frame}


\section{Características de curvas epidemiológicas}
\begin{frame}{Sumário}
\tableofcontents[currentsection]
\end{frame}

\begin{frame}[c]{Características de curvas epidemiológicas}
\begin{figure}
	\centering
	\includegraphics[width=1.00 \linewidth]{figuras/pico}
\end{figure}
\end{frame}

\begin{frame}[c]{Características de curvas epidemiológicas}
\begin{figure}
	\centering
	\includegraphics[width=1.00 \linewidth]{figuras/pico3}
\end{figure}
\end{frame}

\begin{frame}[c]{Características de curvas epidemiológicas}
\vspace{-0.5 cm}
\begin{figure}
	\centering
	\includegraphics[width=0.7 \linewidth]{figuras/ebola}
\end{figure}
\justifying
\footnotesize
\href{https://doi.org/10.1016/j.idm.2019.12.009}{MA, Junling. Estimating epidemic exponential growth rate and basic reproduction number. \textbf{\textcolor{blue}{Infectious Disease Modelling}}, v. 5, p. 129-141, 2020.}
\end{frame}

\begin{frame}[c]{Características de curvas epidemiológicas}
\vspace{-0.5 cm}
\begin{figure}
	\centering
	\includegraphics[width=0.7 \linewidth]{figuras/espanhola}
\end{figure}
\justifying
\footnotesize
\href{https://doi.org/10.1016/j.idm.2019.12.009}{MA, Junling. Estimating epidemic exponential growth rate and basic reproduction number. \textbf{\textcolor{blue}{Infectious Disease Modelling}}, v. 5, p. 129-141, 2020.}
\end{frame}

\section{Análise exploratória dos dados da Covid-19}
\begin{frame}{Sumário}
\tableofcontents[currentsection]
\end{frame}

\begin{frame}[c]{Linha do tempo}
\vspace{-0.3 cm}
\begin{figure}
	\centering
	\hspace{-0.8 cm}
	\includegraphics[width=11.5cm,height=8.3cm]{figuras/timeline.pdf}
\end{figure}
\end{frame}

\begin{frame}[c]{Dados para análises}
\hspace{-1.5 cm}\begin{minipage}[c]{3.0 cm}
	\centering
	\includegraphics[width=0.8 \linewidth]{figuras/johns} 
\end{minipage}
\begin{minipage}[r]{8 cm}
	\scriptsize \textbf{Fonte:} Johns Hopkins University  \\
    \textbf{Dashboard:} \href{https://coronavirus.jhu.edu/map.html}{https://coronavirus.jhu.edu/map.html} \\ 
    \textbf{Dados:} \href{https://data.humdata.org/dataset/novel-coronavirus-2019-ncov-cases}{https://data.humdata.org/dataset/novel-coronavirus-2019-ncov-cases} \\
\end{minipage}
\vspace{0.4 cm}

\hspace{-1.5 cm}\begin{minipage}[c]{3.0 cm}
	\centering
	\includegraphics[width=0.4 \linewidth]{figuras/worldometers} 
\end{minipage}
\begin{minipage}[r]{8 cm}
	\scriptsize \textbf{Fonte:} Worldometer - estatísticas mundiais em tempo real  \\
	\textbf{Site:} \href{www.worldometers.info}{www.worldometers.info} \\ 
	\textbf{Coronavírus:} \href{https://www.worldometers.info/coronavirus/\#countries}{https://www.worldometers.info/coronavirus/\#countries} \\
\end{minipage}
\vspace{0.4 cm}

\hspace{-1.5 cm}\begin{minipage}[c]{3.0 cm}
	\centering
	\includegraphics[width=0.4 \linewidth]{figuras/ms} 
\end{minipage}
\begin{minipage}[r]{8 cm}
	\scriptsize \textbf{Fonte:} Ministério da Saúde  \\
	\textbf{Site:} \href{https://saude.gov.br/}{https://saude.gov.br/} \\ 
	\textbf{Painel Covid-19:} \href{https://covid.saude.gov.br/}{https://covid.saude.gov.br/} \\
\end{minipage}
\vspace{0.4 cm}

\hspace{-1.5 cm}\begin{minipage}[c]{3.0 cm}
	\centering
	\includegraphics[width=0.95 \linewidth]{figuras/ufv} 
\end{minipage}
\begin{minipage}[r]{8 cm}
	\scriptsize \textbf{Fonte:} 
	 Universidade Federal de Viçosa - atualizado conforme boletins das secretarias e Ministério da Saúde.  \\
	\textbf{Site:} \href{https://covid19br.wcota.me/}{https://covid19br.wcota.me/} \\ 
	\textbf{DOI:} \href{https://doi.org/10.1590/SciELOPreprints.362}{https://doi.org/10.1590/SciELOPreprints.362} 
\end{minipage}
\end{frame}


\begin{frame}[t]{Curvas epidemiológicas de alguns países}
\vspace{-0.3 cm}
\begin{figure}
	\hspace{-0.5 cm}
	\includegraphics[width=1.0 \linewidth]{figuras/mundo0}
\end{figure}
\end{frame}

\begin{frame}[c]{Curvas epidemiológicas de alguns países}
\vspace{-0.2 cm}
\begin{figure}
	\hspace{-0.5 cm}
	\includegraphics[width=1.0 \linewidth]{figuras/mundo1}
\end{figure}
\end{frame}

\begin{frame}[c]{Curvas epidemiológicas de alguns países}
\begin{figure}
	\centering
	\includegraphics[width=0.9 \linewidth]{figuras/mundo2}
\end{figure}
\end{frame}

\begin{frame}[c]{Curvas epidemiológicas de alguns países}
\begin{figure}
	\centering
	\includegraphics[width=0.9\linewidth]{figuras/mundo3}
\end{figure}
\end{frame}

\begin{frame}[c]{Curvas epidemiológicas de alguns países}
\begin{figure}
	\centering
	\includegraphics[width=0.9\linewidth]{figuras/mundo4}
\end{figure}
\end{frame}

\begin{frame}[c]{Curvas epidemiológicas de alguns países}
\begin{figure}
	\centering
	\includegraphics[width=0.9\linewidth]{figuras/mundo5}
\end{figure}
\end{frame}

\begin{frame}[c]{Curvas epidemiológicas de alguns países}
\begin{figure}
	\centering
	\includegraphics[width=0.9\linewidth]{figuras/mundo6}
\end{figure}
\end{frame}

\begin{frame}[c]{Curvas epidemiológicas de alguns países}
\begin{figure}
	\centering
	\includegraphics[width=0.9\linewidth]{figuras/mundo7}
\end{figure}
\end{frame}

\begin{frame}[c]{Letalidade e testagem de alguns países}
\begin{figure}
	\centering
	\includegraphics[width=1.0\linewidth]{figuras/mundo8}
\end{figure}
\end{frame}

\begin{frame}[c]{A Covid-19 no Brasil}
\begin{figure}
	\centering
	\includegraphics[width=0.9 \linewidth]{figuras/brasil1}
\end{figure}
\end{frame}

\begin{frame}[c]{A Covid-19 no Brasil}
\begin{figure}
	\centering
	\includegraphics[width=0.9 \linewidth]{figuras/brasil2}
\end{figure}
\end{frame}

\begin{frame}[c]{A Covid-19 no Brasil}
\begin{figure}
	\centering
	\includegraphics[width=0.9 \linewidth]{figuras/brasil3}
\end{figure}
\end{frame}

\begin{frame}[c]{A Covid-19 no Brasil}
\begin{figure}
	\centering
	\includegraphics[width=1.0 \linewidth]{figuras/mapabrasil}
\end{figure}
\end{frame}

\begin{frame}[c]{A Covid-19 na Região Centro-Oeste do Brasil}
\begin{figure}
	\centering
	\includegraphics[width=0.9 \linewidth]{figuras/co1}
\end{figure}
\end{frame}

\begin{frame}[c]{A Covid-19 na Região Centro-Oeste do Brasil}
\begin{figure}
	\centering
	\includegraphics[width=0.9 \linewidth]{figuras/co2}
\end{figure}
\end{frame}

\begin{frame}[c]{A Covid-19 na Região Centro-Oeste do Brasil}
\begin{figure}
	\centering
	\includegraphics[width=0.9 \linewidth]{figuras/co3}
\end{figure}
\end{frame}

\begin{frame}[c]{A Covid-19 na Região Centro-Oeste do Brasil}
\begin{figure}
	\centering
	\includegraphics[width=0.9 \linewidth]{figuras/co4}
\end{figure}
\end{frame}

\begin{frame}[c]{A Covid-19 na Região Centro-Oeste do Brasil}
\begin{figure}
	\centering
	\includegraphics[width=1.0 \linewidth]{figuras/co9}
\end{figure}
\end{frame}

\begin{frame}[c]{A Covid-19 em Goiás}
\begin{figure}
	\centering
	\includegraphics[width=0.9 \linewidth]{figuras/co5}
\end{figure}
\end{frame}

\begin{frame}[c]{A Covid-19 em Goiás}
\begin{figure}
	\centering
	\includegraphics[width=0.9 \linewidth]{figuras/co6}
\end{figure}
\end{frame}

\begin{frame}[c]{A Covid-19 em Goiás}
\begin{figure}
	\centering
	\includegraphics[width=0.9 \linewidth]{figuras/co7}
\end{figure}
\end{frame}

\begin{frame}[c]{A Covid-19 em Goiás}
\begin{figure}
	\centering
	\includegraphics[width=1.0 \linewidth]{figuras/isolamento1}
\end{figure}
\footnotesize
\href{https://mapabrasileirodacovid.inloco.com.br/pt/}{https://mapabrasileirodacovid.inloco.com.br/pt/}
\begin{picture}(10,9)
\footnotesize
\put(20,217){\textcolor{blue}{O ideal é 70\% (OMS)}}
\end{picture}
\end{frame}

\begin{frame}[c]{A Covid-19 em Goiás}
\begin{figure}
	\centering
	\includegraphics[width=1.0 \linewidth]{figuras/isolamento2}
\end{figure}
\footnotesize
\href{https://mapabrasileirodacovid.inloco.com.br/pt/}{https://mapabrasileirodacovid.inloco.com.br/pt/}
\begin{picture}(10,9)
\footnotesize
\put(20,217){\textcolor{blue}{O ideal é 70\% (OMS)}}
\end{picture}
\end{frame}

\begin{frame}[c]{A Covid-19 em Goiás}
\begin{figure}
	\centering
	\includegraphics[width=0.80 \linewidth]{figuras/isolamento3}
\end{figure}
\vspace{-0.2 cm}
\footnotesize
\href{https://mapabrasileirodacovid.inloco.com.br/pt/}{https://mapabrasileirodacovid.inloco.com.br/pt/}
\begin{picture}(10,9)
\footnotesize
\put(20,222){\textcolor{blue}{O ideal é 70\% (OMS)}}
\end{picture}
\end{frame}

\begin{frame}[c]{A Covid-19 em Goiás}
\begin{figure}
	\centering
	\includegraphics[width=1.0 \linewidth]{figuras/dashgoias2}
\end{figure}
\vspace{-0.2 cm}
\footnotesize
\href{https://extranet.saude.go.gov.br/pentaho/api/repos/:coronavirus:paineis:painel.wcdf/generatedContent}{\textcolor{blue}{Dashboard da Covid-19} elaborado pela Secretaria Estadual de Saúde Goiás}
\end{frame}

\begin{frame}[c]{A Covid-19 em Goiás}
\begin{figure}
	\centering
	\includegraphics[width=1.0 \linewidth]{figuras/dashgoias}
\end{figure}
\vspace{-0.2 cm}
\footnotesize
\href{https://extranet.saude.go.gov.br/pentaho/api/repos/:mapa_de_leitos:paineis:painel.wcdf/generatedContent\#dashboardPage}{\textcolor{blue}{Dashboard} da distribuição de leitos nos hospitais do Estado de Goiás}
\end{frame}

\begin{frame}[t]{A Covid-19 em Goiás}
\vspace{-0.3 cm}
\begin{figure}[htb]
	\begin{minipage}[t]{.45\textwidth}
		\centering
		\includegraphics[width=4.5cm,height=3.8cm]{figuras/mapa1a}
	\end{minipage}
	\begin{minipage}[t]{.45\textwidth}
		\hspace{-0.2 cm}
		\centering
		\includegraphics[width=4.5cm,height=3.8cm]{figuras/mapa1b}
	\end{minipage}  
\end{figure}
\vspace{-0.5 cm}
\begin{figure}[htb]
	\begin{minipage}[t]{.45\textwidth}
		\centering
		\includegraphics[width=4.5cm,height=3.8cm]{figuras/mapa1c}
	\end{minipage}
	\begin{minipage}[t]{.45\textwidth}
		\hspace{0.05 cm}
		\centering
		\includegraphics[width=4.9cm,height=3.8cm]{figuras/mapa1d}
	\end{minipage}  
\end{figure}
\begin{picture}(10,9)
\tiny
\put(70,120){\textcolor{blue}{17 municípios (6,9\%)}}
\put(220,120){\textcolor{blue}{58 municípios (23,6\%)}}
\put(70,15){\textcolor{blue}{130 municípios (52,8\%)}}
\put(220,15){\textcolor{blue}{210 municípios (85,4\%)}}
\end{picture}
\end{frame}

\begin{frame}[t]{A Covid-19 em Goiás}
\vspace{-0.3 cm}
\begin{figure}[htb]
	\begin{minipage}[t]{.45\textwidth}
		\centering
		\includegraphics[width=4.5cm,height=3.8cm]{figuras/mapa2a}
	\end{minipage}
	\begin{minipage}[t]{.45\textwidth}
		\hspace{-0.2 cm}
		\centering
		\includegraphics[width=4.5cm,height=3.8cm]{figuras/mapa2b}
	\end{minipage}  
\end{figure}
\vspace{-0.5 cm}
\begin{figure}[htb]
	\begin{minipage}[t]{.45\textwidth}
		\centering
		\includegraphics[width=4.5cm,height=3.8cm]{figuras/mapa2c}
	\end{minipage}
	\begin{minipage}[t]{.45\textwidth}
		\hspace{0.05 cm}
		\centering
		\includegraphics[width=4.9cm,height=3.8cm]{figuras/mapa2d}
	\end{minipage}  
\end{figure}
\begin{picture}(10,9)
\tiny
\put(70,120){\textcolor{blue}{1 município (0,4\%)}}
\put(220,120){\textcolor{blue}{12 municípios (4,9\%)}}
\put(70,15){\textcolor{blue}{41 municípios (16,7\%)}}
\put(220,15){\textcolor{blue}{83 municípios (33,7\%)}}
\end{picture}
\end{frame}

\begin{frame}[c]{A Covid-19 em Goiás}
\begin{figure}[h]
		\centering
		\includegraphics[width=1.0 \linewidth]{figuras/mapa3}
\end{figure}
\end{frame}

\begin{frame}[c]{OBRIGADO!!!}
\hspace{-1.5 cm}\begin{minipage}[c]{3.0 cm}
	\centering
	\includegraphics[width=0.8 \linewidth]{figuras/blog} 
\end{minipage}
\begin{minipage}[r]{8 cm}
	\footnotesize \textbf{Blog:} \href{https://www.thiagovalentim.me/}{https://www.thiagovalentim.me/} 
\end{minipage}
\vspace{0.4 cm}

\hspace{-1.5 cm}\begin{minipage}[c]{3.0 cm}
	\centering
	\includegraphics[width=0.65 \linewidth]{figuras/github} 
\end{minipage}
\begin{minipage}[r]{8 cm}
	\footnotesize \textbf{GitHub:} \href{https://github.com/ThiagoValentimMarques/IFG2020}{https://github.com/ThiagoValentimMarques/IFG2020} 
\end{minipage}
\vspace{0.5 cm}

\hspace{-1.5 cm}\begin{minipage}[c]{3.0 cm}
	\centering
	\includegraphics[width=0.56 \linewidth]{figuras/email} 
\end{minipage}
\begin{minipage}[r]{8 cm}
	\footnotesize \textbf{Endereço eletrônico:} thiago.valentim@ifrn.edu.br
\end{minipage}

\end{frame}


\begin{frame}
\maketitle
\end{frame}


\end{document}